%  -----------------------------------------------------------------------------
%  Modified:October 2024
%  -----------------------------------------------------------------------------
\documentclass{letter} 

 \date{\today}
 
\usepackage{color}

\begin{document}
\begin{letter}{
Chemical Engineering Science\\
Professor Wei Ge, Editor\\}

\opening{Dear Prof. Ge,}

Please find enclosed the revised version of our manuscript (CES-D-24-01867):
``Numerical Simulation of Bubble Deformation and Breakup under 
Simple Linear Shear Flows''.
We thank the reviewers for their comments and thoughts regarding improvement 
of our paper. We believe that we have addressed all of the reviewers’ 
concerns; the changes are itemized in detail below.


\par\noindent
NOTE: all references to equations and figures below are with
reference to the numbering scheme of the revised version of the paper,
not the original version.  Also, changes are hi-lighted in red.
\par\noindent

Changes made in response to comments of Reviewer 4: 
\begin{enumerate}
%%1
\item
\textsf
{The investigated ranges of density ratio ($\lambda$) and viscosity ratio ($\eta$) could be expanded to encompass lower values, which 
would be applicable to various gas-liquid systems, including molten metal processes and slurry bubble columns.}
\vspace{3 mm}

Response: \\
Thank you for your kind comments. 
In the revised paper, we have completely modified the ``Introduction" and stated the target and applicable processes of this study.\\
\\
Please see the ``Introduction".
\\

%%2
\par\noindent
\item
\textsf
{Additional details regarding the computational tools/code and algorithms employed would enhance the reproducibility of the 
results presented in this study.}
\vspace{3 mm}

Response: \\
Thank you very much for your suggestions.  We have additionally referred to our papers (following three papers) 
performed based on the computational code with the same algorithm.

\begin{enumerate}
\item [1.] M. Ohta, T. Furukawa, Y. Yoshida, M. Sussman,\\
A Three-Dimensional Numerical Study on the Dynamics and Deformation of a Bubble Rising in a Hybrid Carreau and FENE-CR Modeled Polymeric Liquid, 
Journal of Non-Newtonian Fluid Mechanics, Vol.265, pp.66-78 (2019).\\
\item [2.] M. Ohta, Y. Akama, Y. Yoshida, \\
M. Sussman, Influence of the Viscosity Ratio on Drop Dynamics and Breakup for a Drop Rising in an Immiscible Low-Viscous Liquid,\\
Journal of Fluid Mechanics, Vol.752, pp.383-409 (2014).\\
\item [3.]  M. Ohta, M. Sussman, \\
The Buoyancy-Driven Motion of a Single Skirted Bubble or Drop Rising through a Viscous Liquid, \\
Physics of Fluids, Vol.24, 112101 (18pp) (2012).\\
\end{enumerate}
Please see lines 223 to 227.
\\

%%3
\par\noindent
\item
\textsf
{The authors should elucidate the rationale behind constraining the viscosity ratio to values below $10^{-3}$.}
\vspace{3 mm}

Response: \\
In the paper, we already described physical properties (density, viscosity, surface tension) set in the computations:\\
\\
Bubble: $\rho_{\rm b}$ = 1.2 kg/m$^{3}$ and $\mu_{\rm b} = 1.8 \times 10^{-5} $ Pa$\cdot$s\\
Liquid:  $\rho_{\rm m}$ = 1000 kg/m$^{3}$ and $\mu_{\rm m}$ = a changeable value corresponding to $Ca$ [Pa$\cdot$s]\\
Surface tension : $\sigma =2.5 \times 10^{-2}$ N/m. \\
Then, $\mu_{\rm m}$ had $2.0 \times 10^{-2} \sim 6.0 \times 10^{-2}$ Pa$\cdot$s and
$V$ had about 1.1 $\sim$ 1.3 m/s for the range of $Ca$ and $Re$ in our computaions.\\
When $\mu_{\rm m}$ = $2.0 \times 10^{-2} \sim 6.0 \times 10^{-2}$ Pa$\cdot$s, we have $\eta < 1.0 \times 10^{-3}$.\\
We have added the range of $\mu_{\rm m}$ and $V$ used in this study.\\
\\
Please see lines 119 to 120.
\\

%%4
\par\noindent
\item
\textsf
{While gravitational effects were not considered in this investigation, a brief discussion of the potential influence 
of the Froude number (Fr) on bubble breakup dynamics would provide valuable context.}
\vspace{3 mm}

Response: \\
We agree with the reviewer's comment.  \\
In our computations, we set $g = 0$ because we wanted to isolate only the effects of the density and viscosity ratios and 
wanted to compare previous studies dealing with the deformation and breakup of a drop with $\lambda =1$ and $\eta = 1$.
When the gravity is considered, we obtain $Fr$ (Froude number) $\left(= \frac{{\it \Gamma}R}{\sqrt {gR}} \right)$ = 1.7 ($Re$ = 93 and $Ca$ = 0.3) 
and 1.9 ($Re = 43$ and $Ca$ = 0.8). \\
Although the values of $Fr$ are not so large, the effect of gravity (bubble rise motion) may not be completely negligible in terms of $Fr$.
However, bubbles for both conditions reach the breakup at $t$ = about 0.5 s. 
Accordingly, it is expected that the effect of gravity (bubble rise motion) can be negligible to the behavior of bubble deformation and breakup 
around critical $Re$ number conditions.
In the revised paper, we have described these explanations.\\
\\
Please see line 95 (the definiton of $Fr$) and lines 123 to 130.
\\


\end{enumerate}

\closing{On behalf of the authors,}
Mitsuhiro Ohta

\end{letter}
\end{document}
