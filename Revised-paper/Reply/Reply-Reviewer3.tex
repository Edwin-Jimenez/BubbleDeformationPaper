%  -----------------------------------------------------------------------------
%  Modified:October 2024
%  -----------------------------------------------------------------------------
\documentclass{letter} 

 \date{\today}
 
\usepackage{color}

\begin{document}
\begin{letter}{
Chemical Engineering Science\\
Professor Wei Ge, Editor\\}

\opening{Dear Prof. Ge,}

Please find enclosed the revised version of our manuscript (CES-D-24-01867):
``Numerical Simulation of Bubble Deformation and Breakup under 
Simple Linear Shear Flows''.
We thank the reviewers for their comments and thoughts regarding improvement 
of our paper. We believe that we have addressed all of the reviewers’ 
concerns; the changes are itemized in detail below.

\par\noindent
NOTE: all references to equations and figures below are with
reference to the numbering scheme of the revised version of the paper,
not the original version.  Also, changes are hi-lighted in red.
\par\noindent

Changes made in response to comments of Reviewer 3: \begin{enumerate}
%%1
\item
\textsf
{The manuscript needs extensive revision for language and grammar.}
\vspace{3 mm}

Response: \\
We carefully checked English writing and grammar.\\

%%2
\item
\textsf
{Discuss more your results.}
\vspace{3 mm}

Response: \\
We additionally developed a discussion for our results.\\
\\
Please see the red characters in the ``Results and Discussion".
\\

%%3
\par\noindent
\item
\textsf
{Improve your figures.}
\vspace{3 mm}

Response: \\
We are confused about the reviewer's comment because the reviewer did not provide which figures to improve.
We carefully checked all the figures and improved Figs. 4, 5, and 10.\\
\\

%%4
\par\noindent
\item
\textsf
{What is the novelty of your study?\\
Discuss about other studies on numerical simulation of bubble deformation. How your study differs?\\
Improve Introduction section.\\}
\vspace{3 mm}

Response: \\
We have referred to and explained previous studies about bubble deformation and breakup in the ``Introduction".
Besides, we have stated the novelty of this study and the difference between this study and previous studies.
In previous experimental studies on the motion of bubble deformation in a simple shear flow, only bubble deformation under very 
low $Re$ number conditions ($Re \ll 1$) have been examined.  
Meanwhile, only a few studies have been reported for the numerical simulation of bubble deformation and breakup in a simple shear flow.
These previous studies mainly examined the dynamics of bubble deformation in a shear flow.
Concerning bubble breakup, Wei et al. (2012) presented only one numerical result for a bubble breakup process under 
the condition of $Ca$ (capillary number) = 35.  
In this study, we first determined the critical Reynolds number ($Re \gg 1$) that leads to bubble breakup for $Ca = 0.3 \sim 1.0$.
This is the largest novelty in this study.
Additionally, our study revealed characteristics that distinguish a drop's deformation and breakup processes from a bubble.\\
In the revised paper, we have completely modified the ``Introduction".  \\
\\
Please see the ``Introduction".
\\

%%4
\par\noindent
\item
\textsf
{The conclusions are not supported by the results and discussion.}
\vspace{3 mm}

Response: \\
We carefully checked The ``Conclusions", and believe the results and discussion properly support our conclusions.\\
We have made a minor alteration to the  ``Conclusions".\\
\\
Please see the ``Conclusions".
\\


\end{enumerate}

\closing{On behalf of the authors,}
Mitsuhiro Ohta

\end{letter}
\end{document}
