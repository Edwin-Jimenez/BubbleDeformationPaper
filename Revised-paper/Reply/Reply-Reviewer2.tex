%  -----------------------------------------------------------------------------
%  Modified:October 2024
%  -----------------------------------------------------------------------------
\documentclass{letter} 

 \date{\today}
 
\usepackage{color}

\begin{document}
\begin{letter}{
Chemical Engineering Science\\
Professor Wei Ge, Editor\\}

\opening{Dear Prof. Ge,}

Please find enclosed the revised version of our manuscript (CES-D-24-01867):
``Numerical Simulation of Bubble Deformation and Breakup under 
Simple Linear Shear Flows''.
We thank the reviewers for their comments and thoughts regarding improvement 
of our paper. We believe that we have addressed all of the reviewers’ 
concerns; the changes are itemized in detail below.

\par\noindent
NOTE: all references to equations and figures below are with
reference to the numbering scheme of the revised version of the paper,
not the original version.  Also, changes are hi-lighted in red.
\par\noindent

Changes made in response to comments of Reviewer 2: \begin{enumerate}
%%1
\item
\textsf
{Why are there several words in italics in the Introduction section and some other parts?}
\vspace{3 mm}

Response: \\
The italic format was used to emphasize words, But the italic format was removed in the revised 
paper because words in italic format are confusing expressions.\\

%%2
\item
\textsf
{Page 2 Line 14: There should be an "and" before "Renardy"}
\vspace{3 mm}

Response: \\
We appreciate the reviewer's suggestion. The description of the applicable part disappeared 
in the revised paper because we have completely modified the ``Introduction".\\
\\
Please see the ``Introduction".
\\

%%3
\par\noindent
\item
\textsf
{The authors mentioned several DNS methods in the Introduction section while choosing the CLSVOF method in this work. What are the reasons?}
\vspace{3 mm}

Response: \\
As explained in the ``Numerical Analysis", the CLSVOF method couples both the level set and VOF methods and can 
take advantage of both methodologies. 
Consequently,  the CLSVOF method is superior to other interface-capturing/interface-tracking methods. 
This is the reason we selected the CLSVOF method. We only referred to several interface-capturing/interface-tracking 
method methods in introducing previous studies.\\
\\

%%4
\par\noindent
\item
\textsf
{Why can the effect of gravity be disregarded in this work?\\}
\vspace{3 mm}

Response: \\
In our computations, we set $g = 0$ because we wanted to isolate only the effects of the density and viscosity ratios and 
wanted to compare previous studies dealing with the deformation and breakup of a drop with $\lambda =1$ and $\eta = 1$. 
Also, in the revised paper, a brief discussion of the potential influence of gravity (Froude number) on bubble breakup dynamics 
has been added according to another reviewer's suggestion.\\
\\
Please see lines 123 to 130.
\\

%%5
\par\noindent
\item
\textsf
{Page 9: The results of D are compared with those in other literature. 
What are the underlying causes of these differences? It is also noticed that different literatures are used for different verification steps.
Why not use the same literature for verification?}
\vspace{3 mm}

Response: \\
Table 1 shows the comparison of the deformation parameter $D$ (Li et al., Physics of Fluids, Vol.12, 269–282, 2000) for a drop as a function of 
$Re$ ($Ca$ = 0.3, $\lambda =1$ and $\eta = 1$).
The $D$ computationally obtained by Li et al. (2000) is regarded as a benchmark problem in this field. First, we compared the $D$ by Li et al. (2000) 
with our computational results. Also, In Fig. 2,  we compared the appearance of drop breakup (Renardy and Cristini, 
Physics of Fluids, Vol.13, 2161–2164, 2001) with our computational results.
On the other hand, regarding quantitative physical quantities on bubble deformation and breakup, only experimental results for $D$ exist under 
the condition of $Re \simeq 0$.
Thus, As shown in Table 2, we performed the comparison of the deformation parameter $D$ for a bubble as a function of  $Ca$ 
($Re \simeq 0$, $\lambda  \simeq 0$ and $\eta  \simeq 0$). In the experiments, obtaining precise experimental data will be difficult because 
the dynamic motion of a bubble in a highly viscous liquid in an experimental device needs to be accurately set.
We don't know the cause of the difference between both experimental results, but our numerical results were close to those of Rust and 
Manga (Journal of Colloid and 
Interface Science, Vol. 249, 476– 480 2002). For your reference, computations for bubble deformation with the condition of $Re \simeq 0$ 
were hard tasks and a very long computational time (one year) was needed. \\
\\

%%6
\par\noindent
\item
\textsf
{Page 14: What are the units of the variable T?.}
\vspace{3 mm}

Response: \\
$T$ is the dimensionless time defined by $T$ =$\mathit{\Gamma} t$. \\
\\
Please confirm line 354.
\\

%%7
\par\noindent
\item
\textsf
{Figure 4: Drop deformation and breakup are compared under conditions with $Re$ = 1.0 and 1.1. When $Re$ = 1.0, the simulation ends at $T$ = 35.0, and it is stated that drops will not break up. When $R$ = 1.1, the drop also does not break up at $T$ = 35.0. It breaks up at $T$ = 56.7. What about also showing the status of drop under $Re$ = 1.0 at $T$ = 56.7? It might be a more reasonable comparison.}
\vspace{3 mm}

Response: \\
We agree with the reviewer's suggestion. We added results at $T$ = 40.1, 50.0, and 56.7 to the drop status for $Re$ = 1.0 in Fig. 4.
As can be seen from new Fig. 4, a drop remains a stable deformed state after $T$ = 35.0. In the revised paper, such a comment was added.\\
\\
Please confirm lines 380 to 381.
\\

%%8
\par\noindent
\item
\textsf
{Figures 5 - 9: I am curious about the way that the selection of the Ts. Why are they not unified in these five figures?.}
\vspace{3 mm}

Response: \\
Regarding Figs. 5 and 6, we compared bubble dynamics for $Re$ = 92 (Fig. 5) with that for $Re$ = 93 (Fig. 5) at almost the same $T$.
But, we found a mistake in Fig. 5: we replaced the figure with $T$ = 17.4 with the figure with $T$ = 12.2.
In the same way, we replaced the figure with $T$ = 24.3 with the figure with $T$ = 17.0 in Fig. 10. \\

Fig. 7 shows the detailed bubble breakup process. Giving the highest priority to show the detailed bubble breakup process, we selected 
results with $T$ independently of $T$ set in Figs. 5 and 6.\\

Fig. 8 shows the shear stress profile around a bubble for two Reynolds numbers ($Re$ = 50 and 93). In Fig. 8, the shear stress for the 
$Re$ = 50 condition was drawn around the bubble after the bubble attained a stable deformed state.
The shear stress for the case of $Re$ = 93 was depicted when the bubble sufficiently elongated ($T$ = 14.9).
We wanted to compare the maximum shear stress value at both $Re$ conditions. \\
In the revised paper, these explanations for Fig. 8 were added.\\

As for Fig. 9, we drew velocity fields around bubbles corresponding to Fig. 7.\\
\\

%%9
\par\noindent
\item
\textsf
{Page 16 Line 395: Why does the bubble not develop the bulb-like shape like the drop?.}
\vspace{3 mm}

Response: \\
In the breakup of the drop with $\lambda =1$ and $\eta = 1$ (Fig. 4), the drop starts to break at the center of the drop without 
the elongation in the first place because the drop easily breaks under the low $Re$ number. 
Consequently, the drop breakup is seemingly developed with a bulb-like shape because the liquid bridge between two mother 
drops is formed in the breakup process.
On the other hand, in the case of the bubble, very large shear forces are required to deform the bubble because $\lambda \simeq 0$ 
and $\eta \simeq 0$.
Thus, the bubble undergoing large shear forces is largely stretched along the direction of the shear flow, and the very long 
elongated bubble is finally formed. 
Accordingly, the bubble breaks up without the bulb-like shape. In the revised paper, we added these considerations.\\
\\
Please see lines 408 to 413.
\\

%%10
\par\noindent
\item
\textsf
{Page 17 Line 424: What are the possible reasons that the bubble breakup process is different from the drop breakup process?}
\vspace{3 mm}

Response: \\
As mentioned above (Answer 9), the effect of $\lambda$ and $\eta$ on the breakup process is large, and especially $\eta$ is 
a dominant factor for the breakup process. 
The shear stress is defined by $\tau = \mu$$\dot{\gamma}$ ($\dot{\gamma}$: the shear-rate (the velocity gradient)).
As is obvious from this equation, since the viscosity of a gas is much less than that of a liquid($\eta \simeq 0$), so a very large velocity 
gradient needs to be achieved to realize $\tau$ that makes the bubble largely deformed.
As a result, the bubble undergoing large shear forces (velocity gradients) is largely stretched along the direction of the shear flow, and
the appearance of the bubble breakup becomes quite different from that of a drop 
Therefore, to be exact, the bubble breakup process is different from the drop breakup with $\lambda = 1$ and $\eta =1$.  
We added "with $\lambda = 1$ and $\eta =1$" to the revised paper.
Ongoingly, we have been examining the effect of $\lambda$ and $\eta$ on drop breakup, and it is confirmed that large shear forces 
are required to deform the drop when $\lambda = 1$ and $\eta \simeq 0$ and the drop breaks up by way of forming the long 
elongated shape. 
Also, we have referred to the effect of the viscosity ratio on the deformation and breakup of the drop in the revised paper.\\
\\
Please see lines 440 to 442.
\\

%%11
\par\noindent
\item
\textsf
{Section 4.4: What about the velocity field of the breaking drop? In the previous text and figures, it does not seem similar to that of the breaking bubble.}
\vspace{3 mm}

Response: \\
Detailed velocity fields of the deforming and/or breaking drop have already been presented in some literature.
So, we have not shown the velocity fields of the deforming and/or breaking drop here.
The behavior of a breakup influences the velocity fields for the drop and the bubble, so the velocity fields between the drop 
and the bubble is not similar.
In the revised paper, we added these explanations.  \\
\\
Please see lines 497 to 501.
\\


%%12
\par\noindent
\item
\textsf
{Is there any correlation that can be summarized between Rec and Ca?}
\vspace{3 mm}

Response: \\
We need to additionally investigate $Re_{\rm c}$ for a wider $Ca$ number condition to consider correlation between $Re_{\rm c}$ and $Ca$,
but we predict that a correlation with $Re_{\rm c}$ as a function of $Ca$ for predicting the bubble breakup can be formulated.\\
\\


\end{enumerate}

\closing{On behalf of the authors,}
Mitsuhiro Ohta

\end{letter}
\end{document}
