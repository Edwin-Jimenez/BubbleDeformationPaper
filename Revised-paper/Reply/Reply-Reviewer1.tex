%  -----------------------------------------------------------------------------
%  Modified:October 2024
%  -----------------------------------------------------------------------------
\documentclass{letter} 

 \date{\today}
 
\usepackage{color}

\begin{document}
\begin{letter}{
Chemical Engineering Science\\
Professor Wei Ge, Editor\\}

\opening{Dear Prof. Ge,}

Please find enclosed the revised version of our manuscript (CES-D-24-01867):
``Numerical Simulation of Bubble Deformation and Breakup under 
Simple Linear Shear Flows''.
We thank the reviewers for their comments and thoughts regarding improvement 
of our paper. We believe that we have addressed all of the reviewers’ 
concerns; the changes are itemized in detail below.

\par\noindent
NOTE: all references to equations and figures below are with
reference to the numbering scheme of the revised version of the paper,
not the original version.  Also, changes are hi-lighted in red.
\par\noindent

Changes made in response to comments of Reviewer 1: \begin{enumerate}
%%1
\item
\textsf
{Abstract, hightlights, and introduction are written in a way that confuses the reader: the text jumps back and forth between bubbles and drops as well as shear flows and other flow configurations. Then it suddenly moves to lift and deformation rather than breakup. Finally a number of random applications are mentioned without any relation to bubble breakup. The authors should clearly state what their contrbution is from the outset. The reader will not go on until the end to figure that out by himself. The use of shorter paragraphs each addressing a single issue is very helpful.}
\vspace{3 mm}

Response: \\
We have gone through the whole paper fixing the grammar and structure.  We believe the message is very clear now. \\
\\

%%2
\item
\textsf
{Method: The use of only 2 levels of grid refinement appears odd. For an accurate representation of the flow near the interface a very fine mesh is typically needed. Only 32 cells over the bubble diameter is hardly sufficient ot get accurate results as corroborated by the mediorce agreement of the presented simulation results with the experimental data used for validation. Two more levels of grid refinement are likely needed to resolved the flow near the bubble while farther away some coarsening may be possible. Optimizing the grid is imperative in a situaltion like the considerd one due to the strong deformation of the bubble which high curvature of the interface and the singular nature of the breakup event. Fully capturing the breakup, which represents a topological singularity, is impossible without a subgrid scale model.}
\vspace{3 mm}

Response: \\
We totally disagree with the referee here.  The resolution used for our results is enough.  The results shown in Figure 3 are very convincing in that if we refined the grid more, the results would hardly change.  In addition, we have the results in Tables I and II which compare our results to other researchers' simulations and experiments for bubble/drop deformation in shear flows. Finally, at the top of section 3.2, we now list a number of references involving bubbles and drops with severe deformation in which our method was validated by way of grid refinement studies and comparison with experiments.  

There is a big gaping hole in the research in which our results fit in: this is the low Capillary number, Moderate to High Reynolds number regime.  In this void, it is impossible to do experiments unless one is in a microgravity environment.  At the same time, numerical algorithms must be designed in order to robustly and accurately predict the tensile strength of a bubble in the low capillary, moderate to high Reynolds number regime, in which ours is the first.  We believe that the results as currently presented, fill the void and that they should be reported now rather than waiting another year.

Perhaps in future work, in which we have more computing power available to us, we can try an extra refinement, albeit, the results are most likely a foregone conclusion in that they will be in close agreement with the results that we already have.  Also as future work, it is inevitable that researchers will be able to run experiments on the space station which corroborate the results of our simulations that we have now.

We have also added ``Remark 1'' and ``Remark 2'' in section 3.1 which makes it clear why adding another level of refinement would delay the publication of our work by another four years, and why sub-scale models are unnnecessary at pinch off for the bubble deformation under simple shear flow problem.  Again, the results that we have now are publishable and will make an impact, it would be fruitless to wait another 4 years.

%%3
\par\noindent
\item
\textsf
{Results: Mostly just pictures are shown from which little is to be learned. The only results of which future use can be expected is the regime map in Fig. 12. Bu this is also only qualitative and based on very few data points. More combinations of Ca and Re should be provided and a quantitative fit to the regime boundary be obtained. Moreover only two values of density and viscosity ratio equal to 1 and ~0.001 are used. An investigation of intermediate values would be straight forward and would allow to provide also models for the influence of these obviously important parameters. In particular it should be clarified whether the difference in behavior is more due to the density ration or more due to the viscosity ratio. Also it needs to be investigated how small these ratios need to be for the gas phase properties to be considered as negligibly small. For the practically important case of air bubbles in water the viscosity ratio is 0.02, an order of magnitude larger than considered in the paper. Whether this can already be considered negligible or not remains unclear.}
\vspace{3 mm}

Response: \\
We totally disagree with this comment too.  We already report many novel and significant results and sufficiently analyze the data both with strategic plots and quantitative tables.  Adding more would only needlessly delay the paper another year.  
\par
We do agree with the referee in that the recommended new simulations would be an excellent inclusion in future papers.  Please see our modified conclusion, last paragraph.



%%4
\par\noindent
\item
\textsf
{The effect of gravity is neglected which potentially obliterates the results. It is not at all clear that the effect of the shear flow can be separated from that of gravity without rendering the results meaningless for any practical application. Even when one wants to do so it needs to be considered that the bubbles will be deformed into an ellipsoidal shape. Then at least the influence of initial condition has to be checked in order to ensure that the results make any sense. \\}
\vspace{3 mm}

Response: \\
We totally disagree that not including gravity ``obliterates'' the results.

At the least, there are a host of strictly microgravity applications in which our research will be of great help.   We list many new applications, including the microgravity applications, in the revised version: please see the first paragraph in the introduction.  All these applications that we list will significantly benefit from our article.  

We have also added a discussion at the end of section 2 (see the ``Remark'') in which we make some predictions as to how the inclusion of gravity would change the results.

Having said that, we do agree that taking into account the gravity effects and initial condition effects will make good material for future work.  Please see the last paragraph in our modified conclusion.

\\

%%5
\par\noindent
\item
\textsf
{Language and writing (examples):\\
line 191 and following: "Timestep $\Delta t$" appears out of any context. So do "Step 1." to "Step 4".\\
line 232: ""bubble deformation in simple linear shear flow" results" is impossible language. \\}
\vspace{3 mm}

Response: \\
We have fixed these items.  Also, we have made many complete ``run-throughs'' of our paper in order to remove all mistakes and awkward sentences.

\end{enumerate}

\closing{On behalf of the authors,}
Mitsuhiro Ohta

\end{letter}
\end{document}
