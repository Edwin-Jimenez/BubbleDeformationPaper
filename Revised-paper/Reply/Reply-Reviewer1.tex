%  -----------------------------------------------------------------------------
%  Modified:October 2024
%  -----------------------------------------------------------------------------
\documentclass{letter} 

 \date{\today}
 
\usepackage{color}

\begin{document}
\begin{letter}{
Chemical Engineering Science\\
Professor Wei Ge, Editor\\}

\opening{Dear Prof. Ge,}

Please find enclosed the revised version of our manuscript (CES-D-24-01867):
``Numerical Simulation of Bubble Deformation and Breakup under 
Simple Linear Shear Flows''.
We thank the reviewers for their comments and thoughts regarding improvement 
of our paper. We believe that we have addressed all of the reviewers’ 
concerns; the changes are itemized in detail below.

\par\noindent
NOTE: all references to equations and figures below are with
reference to the numbering scheme of the revised version of the paper,
not the original version.  Also, changes are hi-lighted in red.
\par\noindent

Changes made in response to comments of Reviewer 1: \begin{enumerate}
%%1
\item
\textsf
{Abstract, hightlights, and introduction are written in a way that confuses the reader: the text jumps back and forth between bubbles and drops as well as shear flows and other flow configurations. Then it suddenly moves to lift and deformation rather than breakup. Finally a number of random applications are mentioned without any relation to bubble breakup. The authors should clearly state what their contrbution is from the outset. The reader will not go on until the end to figure that out by himself. The use of shorter paragraphs each addressing a single issue is very helpful.}
\vspace{3 mm}

Response: \\
In the revision, we made many improvements to the grammar and structure so that the reader can follow in a much more efficient manner. \\
\\

%%2
\item
\textsf
{Method: The use of only 2 levels of grid refinement appears odd. For an accurate representation of the flow near the interface a very fine mesh is typically needed. Only 32 cells over the bubble diameter is hardly sufficient ot get accurate results as corroborated by the mediorce agreement of the presented simulation results with the experimental data used for validation. Two more levels of grid refinement are likely needed to resolved the flow near the bubble while farther away some coarsening may be possible. Optimizing the grid is imperative in a situaltion like the considerd one due to the strong deformation of the bubble which high curvature of the interface and the singular nature of the breakup event. Fully capturing the breakup, which represents a topological singularity, is impossible without a subgrid scale model.}
\vspace{3 mm}

Response: \\
		There are two main problems identified from the referee: (a) the grid refinement study does not demonstrate enough convergence and (b) even if we could refine the grid an infinite number of times, our break-up model is insufficient and whatever findings that we have would be completely different than if we were able to model the break-up process with a molecular scale model.  We disagree with the referee; we have added more references to our article in which the uncertainty in our article is smaller than that in the latest existing work. Also we note that existing articles on bubble and drop breakup due to shear do not take into account sub-scale models either and these articles demonstrate reasonable agreement with experiments (as does our submission).  Note: if we were considering the merging of droplets in which bouncing versus merging can occur, then sub-scale drainage can be important to model.  This issue is not relevent for our article though.  We have added this discussion to the introduction.

Finally, we note that our simulations already take about one year.  The bubble deformation case, is a much more challenging problem than the drop case due to the large density ratio.  Anymore refinement would extend the time of simulation to 8 years using current technology.  It is agreed though that speeding up the algorithm in order to run more efficiently would certainly reduce the uncertainty of our results even more.  This is future work.
%Muller-Fischer comparison

%%3
\par\noindent
\item
\textsf
{Results: Mostly just pictures are shown from which little is to be learned. The only results of which future use can be expected is the regime map in Fig. 12. Bu this is also only qualitative and based on very few data points. More combinations of Ca and Re should be provided and a quantitative fit to the regime boundary be obtained. Moreover only two values of density and viscosity ratio equal to 1 and ~0.001 are used. An investigation of intermediate values would be straight forward and would allow to provide also models for the influence of these obviously important parameters. In particular it should be clarified whether the difference in behavior is more due to the density ration or more due to the viscosity ratio. Also it needs to be investigated how small these ratios need to be for the gas phase properties to be considered as negligibly small. For the practically important case of air bubbles in water the viscosity ratio is 0.02, an order of magnitude larger than considered in the paper. Whether this can already be considered negligible or not remains unclear.}
\vspace{3 mm}

Response: \\
As in our response to the previous comment, we believe it is unreasonable to report more results; specifically each new data point requires an additional year, and additional computers.  As also mentioned in our response to point 2 above, and hilighted in our revised introduction,
our work as it stands now is significant and novel.  We take the referee comments on as a challenge for future work.

We remark that the significance of our article is the fact that we have 
reported for the first time controlled tensile strength properties for a bubble in shear flow.  In addition, we have made the following noteworthy discoveries along the way:
\begin{enumerate}
\item The behavior of the bubble deformation process around critical $Re$ conditions with $Ca$ = 0.3  $\sim$ 1.0 has been presented for the first time.
\item The behavior of the bubble breakup process at critical $Re$ conditions with $Ca$ = 0.3  $\sim$ 1.0 has been presented for the first time.
\item The value of critical $Re$ numbers was found for $Ca$ = 0.3  $\sim$ 1.0 for the first time.
\item  We showed that most previous computational studies overestimated drop deformation and breakup because most previous studies used a domain size with $W$ (width ($y$directional length)) =4$R$. Drop deformation is promoted when using a domain size with $W=4R$. 
\item The difference in the deformation and breakup between the bubble ($\lambda<<1$) and the drop ($\lambda = 1$ and $\eta = 1$) was clearly shown.
\item Shear-induced bubble/drop breakup critical curves were obtained. 
\end{enumerate}

Regarding the effect of $\lambda$ and $\eta$, ongoingly we are examining the effect of $\lambda$ and $\eta$ on drop breakup for following 
combinations of $\lambda$ and $\eta$:

\begin{tabbing}
 \hspace{55mm} \= \hspace{10mm} \kill
 \hspace{5mm} 1. $\lambda$ = 1.0, $\eta$ =1.0 (done) \> 2. $\lambda \simeq 0.0$, $\eta \simeq 0.0$ (this study) \\ 
 \hspace{5mm} 3. $\lambda$ = 1.0, $\eta \simeq 0.0$ (done) \> 4. $\lambda$ = 1.0, $\eta$ =0.01 \\
 \hspace{5mm} 5. $\lambda$ = 1.0, $\eta \simeq 0.1$ \> 6. $\lambda$ = 0.1, $\eta \simeq 0.0$ (done) \\
 \hspace{5mm} 7. $\lambda$ = 0.1, $\eta$ =0.01 \> 8. $\lambda = 0.1$, $\eta$ =0.1  \\
 \hspace{5mm} 9. $\lambda$ = 0.1, $\eta$ =1.0 (done) \> 10. $\lambda$ = 1.0, $\eta$ =10  \\
 \hspace{5mm} 11. $\lambda$ = 1.0, $\eta$ =100 \> 12. $\lambda$ = 1.0, $\eta$ =1000 \\
\end{tabbing}

Since it takes a very long computational time to obtain results for the effect of $\lambda$ and $\eta$ on drop deformation and breakup, 
all computations for these conditions have not been finished. We will also present the fruits of research on the effect of density and 
viscosity ratios in the next papers. \\
\\

%%4
\par\noindent
\item
\textsf
{The effect of gravity is neglected which potentially obliterates the results. It is not at all clear that the effect of the shear flow can be separated from that of gravity without rendering the results meaningless for any practical application. Even when one wants to do so it needs to be considered that the bubbles will be deformed into an ellipsoidal shape. Then at least the influence of initial condition has to be checked in order to ensure that the results make any sense. \\}
\vspace{3 mm}

Response: \\
In our computations, we set $g = 0$ because we wanted to clearly isolate only the effects of the density and viscosity ratios and 
wanted to compare with previous studies dealing with the deformation and breakup of a drop with $\lambda =1$ and $\eta = 1$.
When the gravity is considered, we obtain $Fr$ (Froude number) $\left(= \frac{{\it \Gamma}R}{\sqrt {gR}} \right)$ = 1.7 ($Re$ = 93 and $Ca$ = 0.3) 
and 1.9 ($Re = 43$ and $Ca$ = 0.8). \\
Although the values of $Fr$ are not so large, the effect of gravity (bubble rise motion) may not be completely negligible in terms of $Fr$.
However, bubbles for both conditions reach the breakup at $t$ = about 0.5 s. 
($V$ (moving wall speed) had about 1.1 $\sim$ 1.3 m/s for the range of $Ca$ and $Re$ in our computaions.)
Therefore, the effect of gravity (bubble rise motion) is expected to be negligible when juxtaposed with the measured tensile strength properties of bubble deformation and breakup 
around the critical $Re$ number conditions. Contrary to what the reviewer suggests, we believe that our computational results in which we only focus on the force balance between the driving shear flow and the bubble surface tension will provide fundamental knowledge and findings for practical applications.  In the revised paper, we have described these explanations.\\
\\
Please see lines 133 to 139.
\\

%%5
\par\noindent
\item
\textsf
{Language and writing (examples):\\
line 191 and following: "Timestep $\Delta t$" appears out of any context. So do "Step 1." to "Step 4".\\
line 232: ""bubble deformation in simple linear shear flow" results" is impossible language. \\}
\vspace{3 mm}

Response: \\
We carefully checked the paper and modified poor English writing, expressions, etc.\\

\end{enumerate}

\closing{On behalf of the authors,}
Mitsuhiro Ohta

\end{letter}
\end{document}
