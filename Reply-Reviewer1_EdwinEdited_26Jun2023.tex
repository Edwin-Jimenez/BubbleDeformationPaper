%  -----------------------------------------------------------------------------
%  Modified: 26 June 2023
%  -----------------------------------------------------------------------------
\documentclass{letter} 


 \date{\today}
 

\begin{document}
\begin{letter}{
Physical Review Fluids\\
Professor Eric S. G. Shaqfeh, Editor\\}

\opening{Dear Prof. Shaqfeh,}

Please find enclosed the revised version of our manuscript (FZ10233):
``Numerical Simulation of Bubble Deformation and Breakup under Simple Linear Shear Flows''
We thank the reviewers for their comments and thoughts regarding improvement 
of our paper. We believe that we have addressed all of the reviewers’ 
concerns; the changes are itemized in detail below.

\par\noindent
NOTE: all references to equations and figures below are with
reference to the numbering scheme of the revised version of the paper,
not the original version.  Also, changes are hi-lighted in red.
\par\noindent

%% REVIEWER 1 and REVIEWER 2 COMMENTS HAVE TO BE SEPARATE.
Changes made in response to comments of Reviewer 1: 
\begin{enumerate}
\item

\textsf
{This is an excellent manuscript and I recommend acceptance.\\
The main results are figure 11 where the boundary between bubbles that break up
and those that do not is plotted in the Ca-Re plane. If possible, it would be
interesting it the authors could include also the boundary for drops.  As the
authors point out, the main difference between bubbles and drops are the ratios
of the material properties.  It would be interesting to see in more detail how
the breakup modes and the boundary between breakup and no breakup depends on
those. My guess is that the viscosity ratio is more important than the density
ratio.}
\vspace{5 mm}

Response: \\
We really appreciate your compliment.

We have added a drop critical curve with $\lambda$ (density ratio) =1 and
$\eta$ (viscosity ratio) = 1 to Fig. 11.  Also, we have mentioned the
difference between both critical curves.

Additionally, we have examined the effect of $\lambda$ and $\eta$ on drop
breakup for the following combinations of $\lambda$ and $\eta$:
\\
\begin{tabbing}
 \hspace{55mm} \= \hspace{10mm} \kill
 \hspace{5mm} 1. $\lambda$ = 1.0, $\eta$ =1.0 (done) \> 2. $\lambda \simeq 0.0$, $\eta \simeq 0.0$ (this study) \\ 
 \hspace{5mm} 3. $\lambda$ = 1.0, $\eta \simeq 0.0$ \> 4. $\lambda$ = 1.0, $\eta$ =0.1 \\
 \hspace{5mm} 5. $\lambda$ = 0.1, $\eta$ =1.0 \> 6. $\lambda$ = 0.1, $\eta$ =0.1  \\
 \hspace{5mm} 7. $\lambda$ = 1.0, $\eta$ =100 \> 8. $\lambda$ = 1.0, $\eta$ =1000 \\
\end{tabbing}

Although computations for all these conditions have not been completed, we have
found out that the effect of the viscosity ratio on drop deformation is larger
than that of the density ratio, as the reviewer expected.  We will also present
our research on the effect of density and viscosity ratios on a future journal
paper.
\par\noindent

\end{enumerate}

\closing{On behalf of the authors,}
Mitsuhiro Ohta

\end{letter}
\end{document}
