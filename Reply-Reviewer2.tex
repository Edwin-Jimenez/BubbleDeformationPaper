%  -----------------------------------------------------------------------------
%  Modified: 17 December 2023
%  -----------------------------------------------------------------------------
\documentclass{letter} 


 \date{\today}
 
\usepackage{color}

\begin{document}
\begin{letter}{
Physical Review Fluids\\
Professor Eric S. G. Shaqfeh, Editor\\}

\opening{Dear Prof. Shaqfeh,}

Please find enclosed the revised version of our manuscript (FZ10233):
``Numerical Simulation of Bubble Deformation and Breakup under Simple Linear Shear Flows''
We thank the reviewers for their comments and thoughts regarding improvement 
of our paper. We believe that we have addressed all of the reviewers’ 
concerns; the changes are itemized in detail below.


\par\noindent
NOTE: all references to equations and figures below are with
reference to the numbering scheme of the revised version of the paper,
not the original version.  Also, changes are hi-lighted in red.
\par\noindent

%% REVIEWER 1 and REVIEWER 2 COMMENTS HAVE TO BE SEPARATE.
Changes made in response to comments of Reviewer 2: 
\begin{enumerate}
%%1
\item
\textsf
{Page 3: The CLSVOF method is not an interface tracking one (in such methods
the interface is calculated simultaneously with the velocity and pressure
fields), but an interface capturing method (in such methods the interface is
reconstructed after the velocity field is calculated at each time step).
Although this is a matter of nomenclature, using the wrong name can be
confusing because the interface tracking algorithms, such as ALE, are
completely different from CLSVOF.}
\vspace{3 mm}

Response: \\
We agree with the reviewer. We have changed the ``interface tracking''
wording to ``interface capturing method''.
\\
Please see the 1st line from the top in ``I\hspace{-1.2pt}I\hspace{-1.2pt}I. A. Numerical method and governing equations''.
\\

%%2
\par\noindent
\item
\textsf
{Page 3: The definitions of the dimensionless numbers, Re, and Ca, are not
appropriate.  These numbers are defined here using the properties of air (the
dispersed material), whereas in the related references, which are used for
validation, they are defined using the properties of the continuous phase.
This is not so important in the validation tests where viscosity and density
ratios between the two phases are set equal to one, but it is important for the
new results. In general, the deformation of the inclusion is caused by the
forces exerted on it by the fluid in the matrix, and even more so when the
inclusion is just an air bubble, which has negligible density and viscosity.
Moreover, the capillary number is the ratio of viscous forces over capillarity
and the Reynolds number is the ratio of inertia over viscous forces, and the
viscous forces are negligible for air in comparison to viscous forces in the
liquid.}
\vspace{3 mm}

Response: \\
Thank you very much for catching this.  We have modified the definitions of
$Re$ and $Ca$ accordingly; it was an accident that we had previously 
defined these quantities in terms of the dispersed material.
The computations presented were performed using $Re$ and $Ca$ defined by the
density and the viscosity of the surrounding liquid.
\\
Please see Eq.(2).
\\

%%3
\par\noindent
\item
\textsf
{Page 3: The direction of gravity is not specified. Is it acting along the
z-direction? If so, see point \#9 below.  If not, in what direction does it
act? In other studies, an air bubble was considered under the action of both
gravity and velocity gradient in the same direction, ref. [3] and [4].
Interesting variations arise from the occurrence of a lift force on a
deformable bubble. Again, these additional works need to be cited and
discussed.}
\vspace{3 mm}

Response: \\
We have added the direction of gravity in Fig. 1 ($z$ direction);
albeit, we prescribe zero gravity in our simulations.  The rationale
for setting gravity equal to zero is given in more detail in our response
to item 9.  The purpose of our study is to do controlled
numerical experiments on the effects of the density and 
viscosity ratios.  We can isolate
the dependencies on these two ratios by prescribing zero gravity.  Although
[3] and [4] do not address the transition to break-up, they address a very
much related aspect to bubble motion in shear flow, so now we cite them in
the introduction.
\\

%%4
\par\noindent
\item
\textsf
{Page 4, top: “The singular Heaviside gradient term in the right hand side of
equation (5) is a body force...”. Although this term arises along with gravity
as a body force, in fact, it presumably acts only on the fluid/fluid
interface.}
\vspace{3 mm}

Response: \\
Following to reviewer's comment, we have provided an additional explanation
``The surface tension force expressed by the singular Heaviside gradient term
acts only on the gas-liquid interface.''
\\
Please see the 5th line from the top of Page 4.
\\

%%5
\par\noindent
\item
\textsf
{Page 4: Eq. (7a) determines the evolution of F, but the need to introduce F in
addition to the colour function required by VOF (which is determined by eq.
(7b)) is not discussed. F was not even mentioned before eq. (7). The authors
should also reference previous works where the need for this distinction has
been explained.}
\vspace{3 mm}

Response: \\
We agree with reviewer's comment. We have added details of the CLSVOF method at
the beginning of Section ``A. Numerical method and governing equations.''
\\

%%6
\par\noindent
\item
\textsf
{Page 5: ``The results shown in Figure 2 verify that our numerical approach can
reproduce the same drop breakup behavior presented in [8].'' The comparison is
only qualitative, not quantitative. One can readily see several differences in
all three cases presented.  The comparison could be improved either by
superimposing the plots or even better by extracting numerical values for the
velocity field or the drop shape. The latter may be more involved, which makes
addressing issue \#7 below even more important.}
\vspace{3 mm}

Response: \\
Though there are differences between previous studies and our study, 
we think a major cause of the discrepancies is attributable to the grid size 
(Renardy: $\Delta x =R/8$, our study: Renardy: $\Delta x =R/16$) as we 
mentioned in the paper.
{\color{red}
%As a quantitative comparison (regarding Question \#8), we have newly made a
%comparison for bubble deformation as Table I\hspace{-1.2pt}I. 
We present a new quantitative comparison, in terms of the Taylor deformation
parameter $D$ as a function of $Ca$, in Table I\hspace{-1.2pt}I for bubble
deformation.  Note that it took several months to obtain additional
computational results for bubble deformation.  Comparisons of our results
against past studies (summarized in Table I\hspace{-1.2pt}I) support the
validity of our bubble deformation computations.
%We confirmed the validity of our computations about bubble deformation.
}
\\

%%7
\par\noindent
\item
\textsf
Page 6: It is well-known that interface capturing methods have drawbacks in
terms of accuracy, which in this problem could lead to uncertainty or
inaccuracy concerning the breakup conditions and the related bubble shapes.
Indeed, the topological changes may affect the small characteristic size of the
filament formed by the distorted drop or bubble, when it becomes comparable
with the grid size. For instance, in their study of two bubbles rising in line
(using VOF and Basilisk as a solver), Zang and Magnaudet, ref. [5], found that
refinement up to R/272 (i.e. the local grid size is 272 times smaller than the
bubble radius) is required in the proximity of the interface to properly
capture the topological changes of millimetric bubbles, since only then the
characteristic size of the grid is smaller than the average film thickness in
typical coalescence conditions. Instead, the authors use R/24 local grid size
at the most. Moreover, the Adaptive Mesh Refinement (AMR) technique is usually
employed for 3-4 consecutive levels of refinement, at least. In the present
work, only two levels of refinement have been used, why? The mesh convergence
study seems quite incomplete.  A more complete study with figures to verify it
should be included in an appendix concerning the mesh and time step
independence of the results. The type of time discretization used is not
stated; is it explicit and which algorithm is used?
\vspace{3 mm}

Response: \\
First, we would like to emphasize to the reviewer that very long
computational times are required for drop/bubble deformation and breakup in
shear flow. For bubble simulations, in particular, even longer computational
times are needed because the time step is much smaller than that for the drop.
In our study, time steps smaller than values determined theoretically from
various time-step constraints were used: the time step was $O(10^{-4}$) s for
the drop and $O(10^{-5}$) s for the bubble.  In addition, it is especially difficult to
solve stably two-phase flows with $\lambda$ (density ratio) $\simeq 0$ and
$\eta$ (viscosity ratio) $\simeq 0$. In some cases, computational times
over a few months long were required.  To overcome the problem of very long
computational times, various computational countermeasures were employed in
past studies. For example, in Renardy's study (Phys. Fluids, 2000) a domain
size with $W$ (width ($y$-directional length)) = 2$D$ ($D$: diameter of
drop/bubble) was employed to reduce the number of computational grid
points.  As shown in Table I\hspace{-1.2pt}I\hspace{-1.2pt}I in this study, using a width of $W =
2D$ accentuates drop deformation. Regarding the grid size, relatively coarse
discretizations have been used in recent studies: $\Delta x =R/48$ (Komrakova et al.,
CES, 2015), $\Delta x =2R/15, R/15, R/25, R/30$ (Hernandez and Rangel, Comput.
Fluids, 2017), $\Delta x =R/15$ (Amania et al., CES, 2019), $\Delta x =R/30$
(Zhang et al., CES, 2021).

\medskip
Komrakova et al. (CES, 2015) and Zhang et al. (CES, 2021) used a LBM method for
two-phase flows, Hernandez and Rangel (Comput. Fluid., 2017) used a VOF method
and Amania et al. (CES, 2019) applied a level set method to computations.
Although Komrakova et al. (CES, 2015) set a higher grid resolution ($\Delta x
=R/48$, which corresponds to $R$=48 lattice units in the LBM method), they
performed computations over a quarter of the full domain, which is inappropriate
for high $Re$ conditions.  In the study by Hernandez and Rangel (Comput.
Fluids, 2017), it was shown that the grid resolution should be $\Delta x \leq
R/15$ for approximately a 5\% error. Zhang et al. (CES, 2021) used a higher grid
resolution ($R$=60 lattice units), but the height of the computational domain was
$4R$ for 3D-computations.  Additionally, the LBM method (Komrakova et al.,
CES(2015), Zhang et al., CES(2021)) allows the use of higher grid resolutions
due to shorter computational times that result from not solving the continuity
equation simultaneously; the decreased computational times come at the
expense of diminished accuracy.

Our simulations were performed under stringent computational settings, in terms
of domain sizes and grid resolutions, selected to guarantee a minimum accuracy
in our numerical results.  The critical $Re$ numbers found from the bubble
simulations presented in this study will provide a minimum standard for future
studies.  Anticipating future increased computational resources and the
development of new numerical algorithms, we have added the following comment:
Additional grid refinements may be desirable to perform further accuracy
evaluations for bubble deformation and breakup.  (see the last line at the
section “C. Consideration of domain and grid sizes”, Page 9)

Regarding the refinement level of AMR, the use of 3-4 consecutive refinement levels is not
always preferred.  A small refinement level is rather suitable in terms of
stable computations and we found that our mesh setting for this purpose was
reasonable.

{\color{red}
We briefly described the type of time discretization, the temporal discretization, 
and the algorithm of our numerical method. Also, we have added information 
on our paper as a reference to obtain more details about the numerical method used in 
this study.}
\\

%%8
\par\noindent
\item
\textsf
{\color{red} Several experimental studies concerning bubbles in shear flow
exist (for example, ref. [6]) so alternatively one could compare the numerical
results with experiments and avoid the comparison with other numerical results,
which are older and may not have achieved convergence due to lack of access to
the software and hardware we have today.}
\vspace{3 mm}

Response: \\
{\color{red} 
%As a quantitative comparison, we have newly made a comparison for bubble 
%deformation as Table I\hspace{-1.2pt}I:
We added a new quantitative comparison for bubble deformation in Table
I\hspace{-1.2pt}I to verify the validity of our computational results; in the
table we compare against the experimental results of Muller-Fischer et al.
(Experiments in fluids, 2008) and Rust and Manga (Journal of Colloid and
Interface Science, 2002). The new numerical results for bubble deformation
indicate that our computations are consistent with past studies.
%We believe that the effectivity of our computations could be proved for bubble deformation.
}
\\

%%9
\par\noindent
\item
\textsf
{The following is the most disturbing issue with this study: It is fine to
examine conditions leading to bubble breakup or other flow instabilities, but
the range of parameter values investigated must correspond to existing
materials and achievable flow conditions.  The authors have given the governing
equations in dimensional form and defined dimensionless numbers incorrectly as
stated in point \#2, but also assigned them values that may not be physically
relevant. Moreover, the bubble size is nowhere stated.................\\
Finally, the Reynolds number of the channel in this case would be: $Re_{\rm
channel} \simeq 6000$.  This high value of Re corresponds to turbulence
conditions.................}
\vspace{3 mm}

Response: \\
As we have stated in response to your comment \#2, correct dimensionless
numbers based on the density and the viscosity of the surrounding liquid were
used in real computations.  Regarding physical properties and conditions in our
computations, we will give you an example for the case of $Ca$ = 0.3 and $Re$
=93.\\

Physical properties (a silicone oil has a very similar physical properties with these values.):\\
\quad $\rho_{\rm m}$ (density of liquid) = 1000 kg/m$^3$, $\mu_{\rm m}$ (viscosity of liquid) = 0.2 Pa$\cdot$s,
$\sigma$ (surface tension) = 0.025 N/m, $R$ (bubble radius) = 5.0 mm.\\
From ${\it \Gamma}$ = $2V/H$ = $2V/6R$, \  $V$ = 3${\it \Gamma} R$ = $\frac{3\sigma Ca}{\mu_{\rm m}} $ = 1.1 m/s.
Also, we obtain $Fr$ (Froude number) = $\frac{{\it \Gamma}R}{\sqrt {gR}}$ = 1.7. \\
In this flow system, the effect of gravity may not be completely negligible. However, in our computations, 
we set $g = 0$ because we wanted to clearly isolate only the effects of 
the density and viscosity ratios. 
From another perspective, it can be said that we investigated 
phenomena in space.
At the end of the Problem Description section, we added a detailed explanation about physical properties and conditions.

Also, regarding the channel $Re$ number, we obtain $Re_{\rm channel} =
\frac{\rho_{\rm m}2VH}{\mu_{\rm m}}$ = 3300 for a simple linear shear-flow.  As
expected, $Re_{\rm channel} = 3300$ falls within turbulence conditions for
simple linear shear-flow of a single phase fluid.  It is important to note that
the generation of turbulence is suppressed and turbulent eddy dissipation
becomes large if bubbles exist in fluid flows. However, as far as we know,
there are no conclusive results on turbulence in a simple linear shear-flow
which includes bubbles.  We predict that the generation of turbulence is
considerably suppressed when bubbles occupy a significant portion of the
flow in a channel, as in this study. At present, we believe that our
computations based on the assumption of laminar flows are reasonable and not
unrealistic.\\

\par\noindent
\item
\textsf
{Page 9, Fig. 5 and related discussion: It is quite strange that the bubble
does not reach a final steady shape, but seems to expand, contract and then
expand again. Here, it is even more important to verify that the simulations
have converged (via mesh and time refinement) and to extend them much further
in time to determine if this periodic motion will prevail or finally lead to a
steady state.}
\vspace{3 mm}

Response: \\
{\color{red}  
Regarding Question \#12, we have made a new figure (shear stress profile) for
the case of $Re$ = 50 and $Ca$ = 0.3.  In the case of $Re$ = 50, the bubble
finally settles into a steady deformed shape. Additionally, we observed a
steady deformed bubble at the $Re$ = 70 (not presented in this paper).  As
discussed in detail in the new Section~I\hspace{-1.2pt}V.C, in the limit, as the value of $Re$
approaches the critical $Re_{\rm c}$ condition, the bubble does not maintain
its deformed state but instead alternates in an elongation and contraction
process.  
%It can be expected the bubble will be in a steady deformed state when the 
%condition of $Re$ number is smaller than the critical $Re$ number ($Re_{\rm c}$),  
%whereas the bubble is largely elongated and contracts to a doglegged 
%shape when the $Re$ number is close to $Re_{\rm c}$.\\
} Moreover, the main object of this study is to find $Re_{\rm c}$; we have
plans for examining bubble periodic deformation with expansion and contraction
in future work.  In ongoing and upcoming work, we will examine the effect of
$\lambda$ and $\eta$ on drop breakup for some combinations of $\lambda$ and
$\eta$ as follows: 
\\
\begin{tabbing}
 \hspace{55mm} \= \hspace{10mm} \kill
 \hspace{5mm} 1. $\lambda$ = 1.0, $\eta$ =1.0 (done) \> 2. $\lambda \simeq 0.0$, $\eta \simeq 0.0$ (this study) \\ 
 \hspace{5mm} 3. $\lambda$ = 1.0, $\eta \simeq 0.0$ \> 4. $\lambda$ = 1.0, $\eta$ =0.1 \\
 \hspace{5mm} 5. $\lambda$ = 0.1, $\eta$ =1.0 \> 6. $\lambda$ = 0.1, $\eta$ =0.1  \\
 \hspace{5mm} 7. $\lambda$ = 1.0, $\eta$ =100 \> 8. $\lambda$ = 1.0, $\eta$ =1000 \\
\end{tabbing}
After we complete the study of the effect of $\lambda$ and $\eta$, we will work
on periodic bubble deformation with expansion and contraction.  Due to
prohibitely long computation times, additional computing resources and improved
numerical algorithms are necessary to simulate multiple periodic cycles in
bubble expansion and contraction.\\

\par\noindent
\item
\textsf
{ Page 13: Fig. 11, which distinguishes the two areas of bubble breakup against
just bubble deformation, presents only 4 points and a curve connecting them,
although this is an important physical result. Several more points must be
included in it to reach a definitive conclusion.}
\vspace{3 mm}

Response: \\
We appreciate the reviewer's opinion, but it is important to keep in mind that
at the start of our study there were no guiding initial results that indicated
even potential values for critical Reynolds numbers for a bubble. The critical
Reynolds numbers found and presented were obtained after extensive and
expensive computations.  After lengthy computations, we were able to determine
the bubble $Re_{\rm c}$ number for 4 $Ca$ numbers.  Nevertheless, we believe
that a bubble breakup critical curve drawn based on 4 points will be useful for
low $Ca$ numbers.  Following the suggestion from the reviewer 1, we have added
a drop breakup critical curve ($\lambda$ = $\eta$ =1.0) to Fig. 12 (former Fig. 11).  
The difference in $Re_{\rm c}$ number between the bubble and the drop is clearer in
the new Fig. 12.  But, the reviewer's opinion is appropriate. 
In future work, we will determine the $Re_{\rm c}$ for a variety of high $Ca$ numbers 
to construct a detailed critical curve for a wider range of $Ca$ numbers.\\


\par\noindent
\item
\textsf
{\color{red}  End of conclusions section: ........................
These phrases state the obvious, the densities and viscosities are different between drops and bubbles and, therefore, they cause different dynamics. 
Instead, it is imperative that the authors give the different physics that the property differences generate. 
To this end, they should examine thoroughly and in every detail the velocity and stress fields in each arrangement. 
This may require additional figures and text, but it must be done. Without this examination and its conclusions, this study is more appropriate for a 
Journal on numerical methods in fluid flows than a physical review Journal. This is the second most important issue with this presentation}
\vspace{3 mm}

Response: \\
{\color{red}
We agree with reviewer's comment. In response, we have added two new images of the
shear stress profile for a bubble for two $Re$ conditions in Fig. 8.
A discussion of the new results, in terms of the shear stresses acting on the
bubble at the two different conditions, is given in Section~I\hspace{-1.2pt}V.C. 
}
\\


\end{enumerate}

Less important issues
\begin{enumerate}

\item
\textsf
{Page 2: The symbols for the domain length, width and height are L(x), W(y) and H(z), the variables in parenthesis indicate the direction related 
to the lengths L, W, and H. However, these are geometric parameters, and could be confused with functions, in the way they are given by the 
authors here and elsewhere in the text.\\
}
\vspace{5 mm}

Response: \\
We agree with reviewer's comment. We deleted $(x), (y), (z)$ and will simply
express the domain length, width and height as $L, W, H$.\\

\par\noindent
\item
\textsf
{Page 6: “... depicted in Figure 4(a), uses Ca = 0:3 and Re = 1:1.” It should be Figure 4(b).\\
}
\vspace{5 mm}

Response: \\
We've fixed the error.\\

\par\noindent
\item
\textsf
{Page 8: “... process is the almost same...” should be “... process is almost the same...”.\\
}
\vspace{5 mm}

Response: \\
We've fixed the typo.\\

\par\noindent
\item
\textsf
{The authors could cite another relevant paper where a similar analysis was undertaken [10]. 
The main difference is that the present computations are fully 3D, while in this suggested study shear 
flow until breakup is examined of a 2D bubble. Nevertheless, there are some similarities 
that should be recognized. The present work is more accurate and extensive.\\
}
\vspace{5 mm}

Response: \\
The paper by Wei et al. is now cited in the Introduction.\\

\par\noindent
\item
\textsf
{The authors should use D instead of De as the symbol for the parameter introduced by Taylor to 
determine the extent for bubble deformation, because De stands for the Deborah number.\\
}
\vspace{5 mm}

Response: \\
We have changed the symbol from $De$ to $D$ as suggested.\\

\end{enumerate}

\closing{On behalf of the authors,}
Mitsuhiro Ohta

\end{letter}
\end{document}
